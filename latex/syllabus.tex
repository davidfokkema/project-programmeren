\documentclass[a4paper,11pt]{report}

\usepackage{a4wide}
\usepackage[utf8]{inputenc}
\usepackage[T1]{fontenc}

\usepackage{minted}
\usepackage{fancyvrb}

\VerbatimFootnotes


\title{Programmeren in Processing}
\author{David Fokkema}


\begin{document}

\maketitle

\chapter{De programmeertaal Processing}

\section{De structuur van een programma}

\subsection{Een statisch programma}

Een \emph{statisch} programma is een programma dat niets méér doet dan één
keer iets op het scherm tekenen.  Er beweegt dus verder niets.  Zo'n soort
programma heeft \emph{geen} structuur.  Je kunt meteen beginnen met
programmeren.  Als je bijvoorbeeld een rode ellips wilt tekenen is het
volgende programma voldoende:
\begin{minted}{java}
fill(255, 0, 0);
ellipse(50, 50, 30, 30);
\end{minted}
Dit programma tekent een cirkel op positie \verb|(50, 50)| en straal
\verb|30|.


\subsection{Een dynamisch programma}

Een \emph{dynamisch} programma is een programma waar dingen veranderen.
Er kunnen bijvoorbeeld dingen bewegen (een stuiterend balletje) of het
programma kan reageren op de gebruiker (de muis volgen, reageren op het
toesenbord).  Vrijwel alle programma's zullen dynamisch zijn.  Zo'n
programma heeft twee verplichte onderdelen: een \verb|setup()| en een
\verb|draw()| functie.  In de \verb|setup()| plaats je opdrachten die
alléén aan het begin van het programma hoeven worden uitgevoerd.  Deze
worden ook maar één keer aangeroepen.  Binnen \verb|draw()| plaats je
opdrachten die steeds opnieuw herhaald moeten worden.  De opdracht om een
venster van 600 bij 400 pixels te openen gaat in \verb|setup()|, maar de
opdrachten om het scherm leeg te maken en een cirkel te tekenen gaan in
\verb|draw()|.  Bijvoorbeeld:
\begin{minted}{java}
void setup() {
    size(600, 400);
    // andere opdrachten die maar *een* keer worden uitgevoerd.
}

void draw() {
    background(0);
    ellipse(50, 50, 30, 30);
    // andere oprachten die moeten worden herhaald.
}
\end{minted}


\subsection{Waar je je variabelen definiëert}

Het is het makkelijkst om variabelen helemaal bovenaan het programma te
definiëren.  Ze zijn dan \emph{globaal}, wat betekent dat je ze overal in
het programma kunt gebruiken.  Voor grotere programma's is dit vaak
\emph{niet} handig,\footnote{Grotere programma's met veel variabelen kunnen
bijvoorbeeld per ongeluk elkaars variabelen gaan veranderen.  Dit gebeurt
vaak met variabelen met namen als \verb|x|, wat zowel de $x$-positie van
een bal, een muur of een tennisracket kan zijn.} maar daar maken we ons
nu niet zo druk om.  Je kunt ze dan alvast een waarde meegeven, maar dat
hoeft niet.  Ook kun je meerdere variabelen tegelijk definiëren, door er
een komma tussen te zetten.  In het volgende voorbeeld definiëren we een
aantal \emph{gehele getallen} (\verb|int|):
\begin{minted}{java}
int x;
int y = 0;
int i, j;

void setup() {
// ... rest van setup
}

void draw() {
// ... reset van draw
}
\end{minted}
Soms wil je een startwaarde meegeven die \emph{niet} bovenaan het
programma al bekend is.  Als je bijvoorbeeld een ellipse wilt tekenen in
het midden van het scherm, zou je kunnen zeggen:
\mint{java}|int x = width / 2;| maar helaas, de variabele \verb|width|
heeft pas de goede waarde als je éérst de grootte van het scherm hebt
gekozen!  Je kunt dan het volgende doen:
\begin{minted}{java}
int x, y;

void setup() {
    size(600, 400);
    x = width / 2;
    y = height / 2;
}

void draw() {
    ellipse(x, y, 50, 50);
}
\end{minted}


\section{Variabelen}

Variabelen worden gebruikt om waardes, bijvoorbeeld getallen, in je
programma te gebruiken en te onthouden.  Je kunt bijvoorbeeld een
variabele hebben die de $x$-positie van een bal onthoudt, of de kleur van
een cirkel, of te onthouden of er nu wél of níet op een muisknop is
gedrukt.

\subsection{Soorten variabelen}

In veel programmeertalen moet je je variabelen heel precies gebruiken.
Eérst moet je zeggen dat je een variabele wilt gaan gebruiken, en wat voor
soort variabele het is.  Er zijn verschillende soorten variabelen, waarvan
de belangrijkste: gehele getallen (\verb|int|), kommagetallen
(\verb|float|), waar/niet-waar variabelen (\verb|boolean|) en tekst
(\verb|String|).  Wanneer je rekent met gehele getallen, dan onthoud de
variabele dus \emph{niets} achter de komma!  Een \verb|boolean| is een
variabele die maar twee waardes kan hebben: waar (\verb|true|) of
niet-waar (\verb|false|).  Hieronder voorbeelden van de definitie van
variablen.
\begin{minted}{java}
int x = 50;
float a = 14.5;
boolean gameOver = false;
String gewonnen = "Gefeliciteerd! U heeft gewonnen!"
\end{minted}


\subsection{Namen van variabelen}


\subsection{Veranderen van variabelen}


\end{document}
