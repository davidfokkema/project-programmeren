\documentclass[a4paper,11pt]{report}

\usepackage{a4wide}
\usepackage[utf8]{inputenc}
\usepackage[T1]{fontenc}

\usepackage{minted}


\title{Programmeren in Processing}
\author{David Fokkema}


\begin{document}

\maketitle

\chapter{De programmeertaal Processing}

\section{De structuur van een programma}

\subsection{Een statisch programma}

Een \emph{statisch} programma is een programma dat niets méér doet dan één
keer iets op het scherm tekenen.  Er beweegt dus verder niets.  Zo'n soort
programma heeft \emph{geen} structuur.  Je kunt meteen beginnen met
programmeren.  Als je bijvoorbeeld een rode ellips wilt tekenen is het
volgende programma voldoende:
\begin{minted}{java}
fill(255, 0, 0);
ellipse(50, 50, 30, 30);
\end{minted}
Dit programma tekent een cirkel op positie \verb|(50, 50)| en straal
\verb|30|.


\subsection{Een dynamisch programma}

Een \emph{dynamisch} programma is een programma waar dingen veranderen.
Er kunnen bijvoorbeeld dingen bewegen (een stuiterend balletje) of het
programma kan reageren op de gebruiker (de muis volgen, reageren op het
toesenbord).  Vrijwel alle programma's zullen dynamisch zijn.  Zo'n
programma heeft twee verplichte onderdelen: een \verb|setup()| en een
\verb|draw()| functie.  In de \verb|setup()| plaats je opdrachten die
alléén aan het begin van het programma hoeven worden uitgevoerd.  Deze
worden ook maar één keer aangeroepen.  Binnen \verb|draw()| plaats je
opdrachten die steeds opnieuw herhaald moeten worden.  De opdracht om een
venster van 600 bij 400 pixels te openen gaat in \verb|setup()|, maar de
opdrachten om het scherm leeg te maken en een cirkel te tekenen gaan in
\verb|draw()|.  Bijvoorbeeld:
\begin{minted}{java}
void setup() {
    size(600, 400);
    // andere opdrachten die maar *een* keer worden uitgevoerd.
}

void draw() {
    background(0);
    ellipse(50, 50, 30, 30);
    // andere oprachten die moeten worden herhaald.
}
\end{minted}


\subsection{Variabelen}

Variabelen worden gewoonlijk helemaal bovenaan het programma gedefiniëerd.
Dit is niet verplicht, maar wel heel gebruikelijk.  Het is dus het beste
om je daar aan te houden.  Je kunt ze dan alvast een waarde meegeven, maar
dat hoeft niet.  Ook kun je meerdere variabelen tegelijk definiëren, door
er een komma tussen te zetten.  Bijvoorbeeld:
\begin{minted}{java}
int x;
int y = 0;
int i, j;

void setup() {
// ... rest van programma
\end{minted}


\end{document}
