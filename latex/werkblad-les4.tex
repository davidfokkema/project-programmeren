\documentclass[a4paper,11pt]{article}

\usepackage{a4wide}
\usepackage[utf8]{inputenc}
\usepackage[T1]{fontenc}

\usepackage{minted}
\usepackage{fancyvrb}

\VerbatimFootnotes


\title{Programmeren in Processing --- les 4}
\author{David Fokkema}


\begin{document}

\maketitle

\section{De muis}

In de volgende opdrachten gaan we iets meer met de muis werken.  We doen
dat in de vorm van korte programmaatjes, die vaak \emph{sketches} worden
genoemd.  De positie van de muiscursor wordt automatisch opgeslagen in de
variabelen \verb|mouseX| en \verb|mouseY|.

\begin{enumerate}
\item Maak een sketch die een cirkeltje tekent om de muiscursor.  Bij het
bewegen van de muis moet het cirkeltje meebewegen.
\item Pas de sketch aan zodat er ook een kruis in beeld komt op de positie
van de cursor.  Teken hiervoor een horizontale lijn over het \emph{hele}
scherm en een verticale lijn over het \emph{hele} scherm die elkaar
snijden bij de muiscursor.
\item Teken een klein bolletje op positie $(x, y)$.  Aan het begin van het
programma maak je \verb|x| en \verb|y| gelijk aan 10.
\item Pas de sketch aan zodat het bolletje de muiscursor rustig volgt.  Je
kunt hiervoor de volgende code gebruiken:
\mint{java}|x = x + (mouseX - x) * 0.05;| maar je mag ook zelf iets
bedenken.
\item Als je met de muis klikt, moet het bolletje naar positie (0, 0)
verplaatst worden.  Maak daarvoor een nieuwe functie in je programma
(zoals \verb|setup| en \verb|draw|) met de naam \verb|mouseClicked|.  Deze
functie wordt automatisch aangeroepen als er op een muisknop wordt
gedrukt.  In déze functie zet je de positie van het bolletje op (0, 0).
\item We willen nu de positie van het bolletje \emph{willekeurig} kiezen
als er op de muisknop wordt gedrukt.  Gebruik hiervoor de functie
\verb|random()|.  Zoek deze functie op in de documentatie van Processing.
Lees de \emph{description, syntax en parameters}.  Het bolletje moet
overal op het scherm terecht kunnen komen.
\end{enumerate}


\section{Het toetsenbord}

We kunnen óók met het toetsenbord werken.  We gaan in de volgende
opdrachten kleuren mengen.

\begin{enumerate}
\item Maak een sketch met drie variabelen, \verb|rood|, \verb|groen| en
\verb|blauw|, met de waarde 0.  Maak de achtergrond deze kleur.
\item Zet midden op het scherm de tekst "Hoi".  Je kunt dit doen met de
functie \verb|text()|.  Om te zorgen dat de tekst ook écht in het midden
verschijnt, gebruik de functie \verb|textAlign()|.
\item Zet nu niet "Hoi" op het scherm, maar de waarde van de variabele
\verb|key|.  Druk op verschillende toetsen en kijk wat er gebeurt.
\item We gaan nu kleuren mengen.  Maak een nieuwe functie
\verb|keyPressed()|.  Gebruik in deze functie if-statements om het
volgende te doen: als de toets R (hoofdletter!) wordt ingedrukt, verhoog
je de variabele \verb|rood| met 20.  Als de toets r (kleine letter!) wordt
ingedrukt, \emph{verlaag} de variabele met 20.  Doe dit ook voor G (groen)
en B (blauw).
\end{enumerate}

\end{document}
